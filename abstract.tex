\begin{abstract} 
    Layer-4 load balancers (LB) play a pivotal role in data centers
    by efficiently distributing traffic across a large number of backend
    servers.  To achieve good network performance, LBs should meet two key
    requirements: load distribution fairness and per-connection consistency
    (PCC). The former guarantees that traffic are distributed evenly among
    servers, while the later ensures that packets belonging to the same
    connection are forwarded to the same server.  We find that existing LBs face
    a dilemma: they either sacrifice fairness to prevent PCC violation, which
    fails to meet the key requirements, or adopt complex scheduling mechanisms,
    which is unscalable and increases deployment costs. 
    In this paper, we present Maat, a load balancer that achieves both fairness
    and PCC while seamlessly integrating into existing data center
    infrastructures. Maat introduces a novel scheduling method called the power
    of one random choice. The method greatly enhancing load distribution
    fairness while reducing PCC violations, and hence improves the utilization
    of all available servers. Maat can fully guarantee PCC by utilizing counting
    Bloom filters with negligible memory overhead. We implement Maat on both
    DPDK servers and programmable switches. Our experimental results show that
    the packet processing overhead of Maat is acceptable.  Compared with other
    Layer-4 LBs, Maat improves load distribution fairness by up to 74.42\% and
    reduces flow completion time by 10.58\%, while fully guaranteeing PCC at the
    cost of minimal memory consumption.
\end{abstract}
